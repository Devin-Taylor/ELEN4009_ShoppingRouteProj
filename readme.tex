\documentclass[]{article}
\usepackage{lmodern}
\usepackage{amssymb,amsmath}
\usepackage{ifxetex,ifluatex}
\usepackage{fixltx2e} % provides \textsubscript
\ifnum 0\ifxetex 1\fi\ifluatex 1\fi=0 % if pdftex
  \usepackage[T1]{fontenc}
  \usepackage[utf8]{inputenc}
\else % if luatex or xelatex
  \ifxetex
    \usepackage{mathspec}
  \else
    \usepackage{fontspec}
  \fi
  \defaultfontfeatures{Ligatures=TeX,Scale=MatchLowercase}
\fi
% use upquote if available, for straight quotes in verbatim environments
\IfFileExists{upquote.sty}{\usepackage{upquote}}{}
% use microtype if available
\IfFileExists{microtype.sty}{%
\usepackage{microtype}
\UseMicrotypeSet[protrusion]{basicmath} % disable protrusion for tt fonts
}{}
\usepackage{hyperref}
\hypersetup{unicode=true,
            pdfborder={0 0 0},
            breaklinks=true}
\urlstyle{same}  % don't use monospace font for urls
\IfFileExists{parskip.sty}{%
\usepackage{parskip}
}{% else
\setlength{\parindent}{0pt}
\setlength{\parskip}{6pt plus 2pt minus 1pt}
}
\setlength{\emergencystretch}{3em}  % prevent overfull lines
\providecommand{\tightlist}{%
  \setlength{\itemsep}{0pt}\setlength{\parskip}{0pt}}
\setcounter{secnumdepth}{0}
% Redefines (sub)paragraphs to behave more like sections
\ifx\paragraph\undefined\else
\let\oldparagraph\paragraph
\renewcommand{\paragraph}[1]{\oldparagraph{#1}\mbox{}}
\fi
\ifx\subparagraph\undefined\else
\let\oldsubparagraph\subparagraph
\renewcommand{\subparagraph}[1]{\oldsubparagraph{#1}\mbox{}}
\fi

\date{}

\begin{document}

\section{Shopping Route Recommender}\label{shopping-route-recommender}

\subsection{ELEN 4009 Software Engineering Project
2016}\label{elen-4009-software-engineering-project-2016}

\subsubsection{The projects primary focus is on documenting the process
of designing a software
product}\label{the-projects-primary-focus-is-on-documenting-the-process-of-designing-a-software-product}

\section{}\label{section}

\textbf{Product Summary}

The Shopping Route Recommender is a web application that aims to improve
the general publics day-to-day life. The product proposes to do this by
reducing the associated stresses of shopping. In specific that
application aims to:

\begin{itemize}
\tightlist
\item
  Allow users to add items to a shopping list on an adhoc basis
\item
  Generate a route that optimises based on either of the following user
  inputs:

  \begin{itemize}
  \tightlist
  \item
    Fastest possible route to obtaining all the products
  \item
    Shortest possible route to obtaining all the products
  \item
    Cheapest possible route to obtaining all the products
  \end{itemize}
\end{itemize}

\textbf{Prerequisits}

\begin{itemize}
\tightlist
\item
  Install
  \href{http://www.howtogeek.com/howto/ubuntu/installing-php5-and-apache-on-ubuntu/}{apache2
  server and PHP5}
\item
  Install and setup postgresql

  \begin{itemize}
  \tightlist
  \item
    \texttt{sudo\ apt-get\ install\ postgresql-9.3}
  \item
    \texttt{sudo\ apt-get\ install\ php5-pgsql}
  \end{itemize}
\end{itemize}

\textbf{Setup}

\textbf{NOTE:} If the user has already installed postgresql they may
need to edit the .php files and change the password

\begin{itemize}
\tightlist
\item
  Create a postgresql database called \texttt{srrec}

  \begin{itemize}
  \tightlist
  \item
    \texttt{sudo\ -i\ -u\ postgres}
  \item
    \texttt{createuser\ -\/-interactive}

    \begin{itemize}
    \tightlist
    \item
      username: \texttt{srrec}
    \end{itemize}
  \item
    \texttt{psql\ postgres}
  \item
    \texttt{\textbackslash{}password\ postgres}

    \begin{itemize}
    \tightlist
    \item
      password: \texttt{srrec}
    \end{itemize}
  \item
    \texttt{ctrl\ +\ d}
  \item
    \texttt{createdb\ srrec}
  \item
    \texttt{psql\ srrec}
  \item
    \texttt{\textbackslash{}password\ srrec}

    \begin{itemize}
    \tightlist
    \item
      password: \texttt{srrec}
    \end{itemize}
  \end{itemize}
\end{itemize}

\textbf{Running the back-end code}

Copy the .txt files located in \texttt{Code/back\_end/} into the root
folder of postgresql, the default for this on Linux is
\texttt{/var/lib/postgresql/9.3/main} \textgreater{}
\texttt{sudo\ cp\ -f\ *.txt\ /var/lib/postgresql/9.3/main}

\textbf{NOTE:} The following steps are temporary implementations until
properly integrated with the front-end

Create the database tables - from within the /Code/back\_end/ folder
run: \textgreater{} \texttt{php\ setup\_database.php}

Create the possible permutations for the shopping list saved on the
database, run: \textgreater{} \texttt{php\ database\_permutations.php}

Create 2D array containing all locations for different routes (each
route is a row and each column in a row is a waypoint), run:
\textgreater{} \texttt{php\ get\_route\_information.php}

\textbf{Running the front-end code}

Copy the contents of the /Code/front\_end/ into the root folder of the
apache2 server, the default location for this in Ubuntu is:
\texttt{/var/www/html/}

Open your web-browser and access \texttt{localhost/index.php} - All
other web pages will be accessible from this page

\end{document}
