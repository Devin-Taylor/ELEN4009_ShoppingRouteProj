\documentclass[10pt, a4paper, onecolumn]{scrartcl}
\usepackage{cite}  
\usepackage{times}
\usepackage{amsmath}
\usepackage{amsfonts}
\usepackage{amssymb}
\usepackage{graphicx}
\usepackage{listings}
\usepackage{enumitem} % used for list - no spaces between items
\usepackage[english]{babel} % English language/hyphenation
\usepackage[top=2.5cm, bottom=3.2cm, left=2cm, right=2cm, columnsep=0.6cm]{geometry}
\usepackage{color} %red, green, blue, yellow, cyan, magenta, black, white
\definecolor{mygreen}{RGB}{28,172,0} % color values Red, Green, Blue
\definecolor{mylilas}{RGB}{170,55,241}

\usepackage{fancyhdr}
\pagestyle{fancyplain}
\fancyhead{}
\renewcommand{\headrulewidth}{0pt} % Remove header underlines
\fancyfoot[L]{} % Empty left footer
\fancyfoot[C]{} % Empty center footer
\fancyfoot[R]{\thepage} 

\usepackage{tikz}
\usetikzlibrary{shapes.geometric,arrows}

\usepackage{sectsty} % Allows customizing section commands
\sectionfont{\centering\normalsize\textbf}
\subsectionfont{\flushleft\normalsize\normalfont\textit}
\subsubsectionfont{\flushleft\normalsize\normalfont\textit}
%\allsectionsfont{\centering} % Make all sections centered

\usepackage[toc,page]{appendix}

\usepackage{pgfgantt}
\usepackage{pgfplots, pgfplotstable}
\pgfplotsset{width=9cm,compat=1.9}


\setlength\parindent{0pt} % remove all indentations in document

%-------------------------------------------------------------------------------------------------------------------------------------------------------------------------------------%
%	                                                                           BEGIN DOCUMENT
%-------------------------------------------------------------------------------------------------------------------------------------------------------------------------------------%

\begin{document}
	
	\section{Shopping Route Recommender}\label{shopping-route-recommender}
	
%		\subsection{ELEN 4009 Software Engineering Project
%			2016}\label{elen-4009-software-engineering-project-2016}
%		
%		\subsubsection{The projects primary focus is on documenting the process
%			of designing a software
%			product}\label{the-projects-primary-focus-is-on-documenting-the-process-of-designing-a-software-product}
	
	\textbf{Product Summary}
	
	The Shopping Route Recommender is a web application that aims to improve
	the general publics day-to-day life. The product proposes to do this by
	reducing the associated stresses of shopping. In specific that
	application aims to:
	
	\begin{itemize}
		\item
		Allow users to add items to a shopping list on an adhoc basis
		\item
		Generate a route that optimises based on either of the following user
		inputs:
		
		\begin{itemize}
			\item
			Fastest possible route to obtaining all the products
			\item
			Shortest possible route to obtaining all the products
			\item
			Cheapest possible route to obtaining all the products
		\end{itemize}
	\end{itemize}
	
	\textbf{Prerequisits}
	
	\begin{itemize}[noitemsep]
		\item
		Install
		http://www.howtogeek.com/howto/ubuntu/installing-php5-and-apache-on-ubuntu/
		\item
		Install and setup postgresql
		
		\begin{itemize}[noitemsep]
			\item
			\texttt{sudo\ apt-get\ install\ postgresql-9.3}
			\item
			\texttt{sudo\ apt-get\ install\ php5-pgsql}
		\end{itemize}
	\end{itemize}
	
	\textbf{Setup}
	
	\textbf{NOTE:} If the user has already installed postgresql they may
	need to edit the .php files and change the password
	
	\begin{itemize}[noitemsep]
		\item
		Create a postgresql database called \texttt{srrec}
		
		\begin{itemize}[noitemsep]
			\item
			\texttt{sudo\ -i\ -u\ postgres}
			\item
			\texttt{createuser\ -\/-interactive}
			
			\begin{itemize}[noitemsep]
				\item
				username: \texttt{srrec}
			\end{itemize}
			\item
			\texttt{psql\ postgres}
			\item
			\texttt{\textbackslash{}password\ postgres}
			
			\begin{itemize}[noitemsep]
				\item
				password: \texttt{srrec}
			\end{itemize}
			\item
			\texttt{ctrl\ +\ d}
			\item
			\texttt{createdb\ srrec}
			\item
			\texttt{psql\ srrec}
			\item
			\texttt{\textbackslash{}password\ srrec}
			
			\begin{itemize}[noitemsep]
				\item
				password: \texttt{srrec}
			\end{itemize}
		\end{itemize}
	\end{itemize}
	
	\textbf{Running the back-end code}
	
	Copy the .txt files located in \texttt{Code/back\_end/} into the root
	folder of postgresql, the default for this on Linux is
	\texttt{/var/lib/postgresql/9.3/main} \textgreater{}
	\texttt{sudo\ cp\ -f\ *.txt\ /var/lib/postgresql/9.3/main}
	
	\textbf{NOTE:} The following steps are temporary implementations until
	properly integrated with the front-end
	
	Create the database tables - from within the /Code/back\_end/ folder
	run: \textgreater{} \texttt{php\ setup\_database.php}
	
	Create the possible permutations for the shopping list saved on the
	database, run: \textgreater{} \texttt{php\ database\_permutations.php}
	
	Create 2D array containing all locations for different routes (each
	route is a row and each column in a row is a waypoint), run:
	\textgreater{} \texttt{php\ get\_route\_information.php}
	
	\textbf{Running the front-end code}
	
	Copy the contents of the /Code/front\_end/ into the root folder of the
	apache2 server, the default location for this in Ubuntu is:
	\texttt{/var/www/html/}
	
	Open your web-browser and access \texttt{localhost/index.php} - All
	other web pages will be accessible from this page
	
\end{document}
