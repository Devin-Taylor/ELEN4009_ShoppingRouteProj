\documentclass[10pt, a4paper, onecolumn]{scrartcl}
\usepackage{cite}  
\usepackage{times}
\usepackage{amsmath}
\usepackage{amsfonts}
\usepackage{amssymb}
\usepackage{graphicx}
\usepackage{listings}
\usepackage{enumitem} % used for list - no spaces between items
\usepackage[english]{babel} % English language/hyphenation
\usepackage[top=2.5cm, bottom=3.2cm, left=2cm, right=2cm, columnsep=0.6cm]{geometry}
\usepackage{color} %red, green, blue, yellow, cyan, magenta, black, white
\definecolor{mygreen}{RGB}{28,172,0} % color values Red, Green, Blue
\definecolor{mylilas}{RGB}{170,55,241}

\usepackage{fancyhdr}
\pagestyle{fancyplain}
\fancyhead{}
\renewcommand{\headrulewidth}{0pt} % Remove header underlines
\fancyfoot[L]{} % Empty left footer
\fancyfoot[C]{} % Empty center footer
\fancyfoot[R]{\thepage} 

\usepackage{tikz}
\usetikzlibrary{shapes.geometric,arrows}

\usepackage{sectsty} % Allows customizing section commands
\sectionfont{\centering\normalsize\textbf}
\subsectionfont{\flushleft\normalsize\normalfont\textit}
\subsubsectionfont{\flushleft\normalsize\normalfont\textit}
%\allsectionsfont{\centering} % Make all sections centered

\usepackage[toc,page]{appendix}

\usepackage{pgfgantt}
\usepackage{pgfplots, pgfplotstable}
\pgfplotsset{width=9cm,compat=1.9}

\setlength\parindent{0pt} % remove all indentations in document

\usepackage{lmodern}
\usepackage{amssymb,amsmath}
\usepackage{ifxetex,ifluatex}
\usepackage{fixltx2e} % provides \textsubscript
\ifnum 0\ifxetex 1\fi\ifluatex 1\fi=0 % if pdftex
\usepackage[T1]{fontenc}
\usepackage[utf8]{inputenc}
\else % if luatex or xelatex
\ifxetex
\usepackage{mathspec}
\else
\usepackage{fontspec}
\fi
\defaultfontfeatures{Ligatures=TeX,Scale=MatchLowercase}
\fi
% use upquote if available, for straight quotes in verbatim environments
\IfFileExists{upquote.sty}{\usepackage{upquote}}{}
% use microtype if available
\IfFileExists{microtype.sty}{%
	\usepackage{microtype}
	\UseMicrotypeSet[protrusion]{basicmath} % disable protrusion for tt fonts
}{}
\usepackage{hyperref}
\hypersetup{unicode=true,
	pdfborder={0 0 0},
	breaklinks=true}
\urlstyle{same}  % don't use monospace font for urls

\providecommand{\tightlist}{%
	\setlength{\itemsep}{0pt}\setlength{\parskip}{0pt}}


%-------------------------------------------------------------------------------------------------------------------------------------------------------------------------------------%
%	                                                                           BEGIN DOCUMENT
%-------------------------------------------------------------------------------------------------------------------------------------------------------------------------------------%
\begin{document}
	
	\newcommand{\horrule}[1]{\rule{\linewidth}{#1}}
	
	\title{\normalfont \normalsize
		\textsc{University of Witwatersrand, Department of Electrical Engineering} \\ [10pt]
		\horrule{0.5pt} \\ [10pt]
		\huge Shopping Route Recommender Installations and Setups \\
		\horrule{2pt} \\ [10pt]}
	\author{\textbf{\normalsize{Luka Cakic (671913), Ronen Freeman (386910), Devin Taylor (603956) and Matthew Marsden (609293)}} \\ [10pt]}
	\date {\normalsize \today}
	
	\maketitle
	
%	\section{Shopping Route Recommender}\label{shopping-route-recommender}
	
%		\subsection{ELEN 4009 Software Engineering Project
%			2016}\label{elen-4009-software-engineering-project-2016}
%		
%		\subsubsection{The projects primary focus is on documenting the process
%			of designing a software
%			product}\label{the-projects-primary-focus-is-on-documenting-the-process-of-designing-a-software-product}

	\textbf{NOTE:} \\
	
	The README file provided on the GitHub repository provides the same information as that below. The README file is maintained on a regular basis which this document is only updated periodically. It is therefore recommended that the user read the README file if possible.\\
	
	\section{Prerequisites}
	
	\begin{itemize}
		\tightlist
		\item
		Install
		\href{http://www.howtogeek.com/howto/ubuntu/installing-php5-and-apache-on-ubuntu/}{apache2
			server and PHP5}
		\item
		Install and setup postgresql
		
		\begin{itemize}
			\tightlist
			\item
			\texttt{sudo\ apt-get\ install\ postgresql-9.3}
			\item
			\texttt{sudo\ apt-get\ install\ php5-pgsql}
		\end{itemize}
	\end{itemize}
	
	\section{Folder Structure (from master branch)}
	
	\begin{itemize}
		\item ROOT
		\begin{itemize}
			\item Code
			\begin{itemize}
				\item back\_end: all back end associated code
				\item css: all css styles used in the front end
				\item front\_end: all front end associated code
				\item images: all images used in the front end
				\item temp: temporary directory - not relevant to project
				\item test\_suite: all test code for the design
			\end{itemize}
			\item Documentation
			\begin{itemize}
				\item class\_modules: description of different class modules
				\item external\_software: description of external software modules utilised
				\item final\_report: overall final report for system
				\item images: all images used in the reports
				\item installations\_and\_initialisations: instructions on how to install the required software
				\item life\_cycle\_motivation: motivation for the selection of SCRUM and all SCRUM documentation
				\item proposal: designed systems proposal
				\item software\_design: software design documentation
				\item srs: software requirement specification
				\item test\_report: description of all tests conducted on the presented prototype
			\end{itemize}
		\end{itemize}
	\end{itemize}
	
	\section{Set-up}
	
	\textbf{NOTE:} If the user has already installed postgresql they may
	need to edit the .php files and change the password
	
	\begin{itemize}
		\tightlist
		\item
		Create a postgresql database called \texttt{srrec}
		
		\begin{itemize}
			\tightlist
			\item
			\texttt{sudo\ -i\ -u\ postgres}
			\item
			\texttt{createuser\ -\/-interactive}
			
			\begin{itemize}
				\tightlist
				\item
				username: \texttt{srrec}
			\end{itemize}
			\item
			\texttt{psql\ postgres}
			\item
			\texttt{\textbackslash{}password\ postgres}
			
			\begin{itemize}
				\tightlist
				\item
				password: \texttt{srrec}
			\end{itemize}
			\item
			\texttt{ctrl\ +\ d}
			\item
			\texttt{createdb\ srrec}
			\item
			\texttt{psql\ srrec}
			\item
			\texttt{\textbackslash{}password\ srrec}
			
			\begin{itemize}
				\tightlist
				\item
				password: \texttt{srrec}
			\end{itemize}
		\end{itemize}
	\end{itemize}
	
	\section{Recommendation}
	
	It is recommended that the user run the file \texttt{back\_end.php} in
	\texttt{Code/test\_suite/back\_end/} to check if the database is configured
	correctly
	
	\section{Running The Back-End Code}
	
	Copy the .txt files located in \texttt{Code/back\_end/} into the root
	folder of postgresql, the default for this on Ubuntu is
	\texttt{/var/lib/postgresql/9.3/main}: \\
	\texttt{sudo\ cp\ -f\ *.txt\ /var/lib/postgresql/9.3/main}\\
	
	Create the database tables - from within the /Code/back\_end/ folder
	run:\\ \texttt{php\ setup\_database.php} \\
	
	\textbf{NOTE:} The following step is a temporary implementation until
	properly integrated with the front-end
	
	Create the possible permutations for the shopping list saved on the
	database, run: \\ \texttt{php\ database\_permutations.php}
	
	\section{Running The Front-End Code}
	
	Copy the contents of the /Code/front\_end/ as well as the \texttt{/css/}
	and \texttt{/images/} folders into the root folder of the apache2
	server, the default location for this in Ubuntu is:
	\texttt{/var/www/html/}\\ \\ \textbf{NOTE:} The css and images must remain
	as folders in the \texttt{/var/www/html} folder: \\
	\texttt{cp\ -r\ css/.\ /var/www/html} and the same for images \\
	
	Open your web-browser and access \texttt{localhost/login.php} - All
	other web pages will be accessible from this page - The default login
	details are: \\ username: \texttt{admin@srrec.com} \\ password:
	\texttt{admin123} \\ \\ Alternatively the user may navigate to the Create
	Account page using the navigation menu of the left hand side and create
	their own login details \\ \\ Clicking login will take you to the
	\texttt{index.php} page which allows the user to add and remove items
	from the shopping list stored on the database \\ \\ Selecting generate route
	defaults to generating the cheapest route and you will be redirected to
	a map which displays this route and corresponding directions.\\
	
	\textbf{NOTE:} In order to see a meaningful output on the map it is
	recommended that the user leave the shopping list items as
	\texttt{SHIRT,\ TV,\ MILK}, this is due to the small databases created
	for prototying purposes.

	

	
\end{document}
