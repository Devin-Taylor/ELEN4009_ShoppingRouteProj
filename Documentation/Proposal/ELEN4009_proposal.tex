\documentclass[10pt, a4paper, onecolumn]{scrartcl}
\usepackage{cite}  
\usepackage{times}
\usepackage{amsmath}
\usepackage{amsfonts}
\usepackage{amssymb}
\usepackage{graphicx}
\usepackage{listings}
\usepackage{enumitem} % used for list - no spaces between items
\usepackage[english]{babel} % English language/hyphenation
\usepackage[top=2cm, bottom= 3.2cm, left=2cm, right=2cm, columnsep=0.6cm]{geometry}
\usepackage{color} %red, green, blue, yellow, cyan, magenta, black, white
\definecolor{mygreen}{RGB}{28,172,0} % color values Red, Green, Blue
\definecolor{mylilas}{RGB}{170,55,241}
\usepackage{fancyhdr}
\pagestyle{fancyplain}
\fancyhead{}
\renewcommand{\headrulewidth}{0pt} % Remove header underlines
\fancyfoot[L]{} % Empty left footer
\fancyfoot[C]{} % Empty center footer
\fancyfoot[R]{\thepage} 
\usepackage{tikz}
\usetikzlibrary{shapes.geometric,arrows}


\usepackage{sectsty} % Allows customizing section commands
\sectionfont{\centering\normalsize\textbf}
\subsectionfont{\flushleft\normalsize\normalfont\textit}
%\allsectionsfont{\centering} % Make all sections centered

\setlength\parindent{0pt} % remove all indentations in document

%----------------------------------------------------------------------------------------
%	BEGIN DOCUMENT
%----------------------------------------------------------------------------------------
\newcommand{\horrule}[1]{\rule{\linewidth}{#1}}

\begin{document}

		\title{\normalfont \normalsize
			\textsc{University of Witwatersrand, Department of Electrical Engineering} \\ [10pt]
			\horrule{0.5pt} \\ [10pt]
			\huge ELEN4009 Shopping Route Recommender \\
			\horrule{2pt} \\ [10pt]}
		\author{\textbf{\normalsize{Luka Cakic (671913), Ronen Freeman (386910) Devin Taylor (603956) and Matthew Marsden (609293)}} \\ [10pt]}
		\date {\normalsize \today}
		
		\maketitle
		
%\makeatletter
%\renewcommand{\maketitle}{\bgroup\setlength{\parindent}{0pt}
%	\begin{flushleft}
%		\textbf{\@title}
%		
%		\@author\vspace{1mm}
%		
%%		\textit{\small{School of Electrical \& Information Engineering, University of the Witwatersrand, Private Bag 3, 2050, Johannesburg, South Africa}} \vspace{10mm} 
%	\end{flushleft}\egroup
%}
%\makeatother
%
%		\twocolumn[
%		\begin{@twocolumnfalse}
%			\title{\normalsize\textbf{ELEN4009: SOFTWARE ENGINEERING}\\[10pt]{PROJECT PROPOSAL - SHOPPING ROUTE RECOMMENDER}\\[10pt]}
%			\author{\textbf{\normalsize{Luka Cakic (671913), Ronen Freeman (386910) Devin Taylor (603956) and Matthew Marsden (609293)}}}
%			\maketitle
%			
%		\end{@twocolumnfalse} ]
%	
%\onecolumn
		
	\section{Project Definition}
		
		The Shopping Route Recommender is a wbe application system that allows a shopper to determine the nearest shops which stock items from his/her shopping list. The shopper interacts with the application by adding items to a shopping list, selecting a preferred optimisation for generating a specific route, and the location within which they want to shop. Once the shopper has added all required entries he/she can run the program to determine a route to follow to purchase all the items added to the shopping list. The three optimisations the program can be set to follow are the shortest route or travel distance, shortest route with minimal total expenses or shortest travel time which allows the shopper to obtain all items on his/her list. 
				
	\section{Front End}
		
		The front end or the GUI of the application will be the user interface of the system, and therefore the means through which the user will interact with the program. It will be user friendly, simple to use and flexible. The user will enter his or her items manually one at a time. Furthermore, the application will give options to save collections of items as a shopping list. Loading of a saved shopping list will be a third method of system input. Further options will be displayed such that the user may specify their preferable means of optimisation, these being: shortest route, minimum travel time and least total expenses. Following the user selection of a 'Generate Route' button, the system will output the optimised route depending on the user selections. This route will be displayed to the user via two methods, one being a list of directions and the other a generated map highlighting the optimised route. Alternative route options will be displayed in a grey route marking colour scheme. \\
		
		The SRR application will also contain a drop down or slide out menu that will contain basic menu entries. These include, but is not limited to, "About", "Settings", "History" and "Help" tabs. The "About" tab would provide information about the application and what the application's intended purpose and functionality is. The "Settings" tab would provide application configuration settings, such as distance metrics (km or miles) and language settings. The "History" would display a list of recently generated shopping routes and their corresponding optimisations. The "Help" tab would provide a Frequently-Asked-Questions (FAQ) section, as well descriptions for the use of the application. The "Help" function is intended to assist a user who would not otherwise know how to use the application. These menu tabs would ensure the application can be a globally used application.  \\
		
		The Back End will return location co-ordinates corresponding to the shops that stock the requested item and that fall within the selected optimisation bracket. The GUI will generate a Google Maps display with an optimised route drawn between shop markers. It is this map the user will use to ensure a good shopping experience and piece of mind.  
	
	\section{Back End}
	
		The Back End consists of all the application processing. The main feature of the Back End will be the database. A Relational Database Management System (RDBMS) will be used to store all web application appropriate information. This type of database is based on a relational model such that tables and fields are interconnected in a logical and sequential manner. PostGre Structured Query Language (PostgreSQL) is the language that will be used in the application to handle the information stored in the RDBMS. \\
	
		Based on the input of the user, the information will be processed at the Back End using PostgreSQL. The database will be structured in such a way that it contains a number of different tables, each interlinked in a relational manner. The tables will include headings such as Items, Shops and Location. The tables will contain their appropriate fields and the relational database scheme will be used to link the tables and fields accordingly. The PostgreSQL implementation will pull out the appropriate location information from the tables based on the users chosen optimisation and current location. The extracted data will be passed onto the Front End. The Front End will generate a Google Maps view and route based on the returned location co-ordinates. The route generated will be based on the preferred optimisation specified by the user, the best route will be displayed on the GUI map display. 
	
	\section{Group Responsibilities}
	
		The final project will be sectioned into two parts. One group pair will work on and perfect the Front End of the application, the other group pair will design the Back End of the overall system. The two Ends will be combined to form the overall user friendly Shopping Route Recommender Program. \\
		
		Ronen Freeman and Luka Cakic will be concentrating on the Front End of the web application. This will involve designing the GUI, receiving Back End data related to the optimised route and generating the Google Maps route. \\
		
		Devin Taylor and Matthew Marsden will be focusing on the Back End. This involves extracting the relational data  and returning the location based co-ordinates to the Front End. The Back End system will be designed as an RDBMS using PostgreSQL.   
	

%----------------------------------------------------------------------------------------
%	REFERENCES
%----------------------------------------------------------------------------------------
	
		
		
\end{document}