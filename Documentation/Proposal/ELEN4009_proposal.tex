\documentclass[10pt, a4paper, onecolumn]{scrartcl}
\usepackage{cite}  
\usepackage{times}
\usepackage{amsmath}
\usepackage{amsfonts}
\usepackage{amssymb}
\usepackage{graphicx}
\usepackage{listings}
\usepackage{enumitem} % used for list - no spaces between items
\usepackage[english]{babel} % English language/hyphenation
\usepackage[top=2cm, bottom= 3.2cm, left=2cm, right=2cm, columnsep=0.6cm]{geometry}
\usepackage{color} %red, green, blue, yellow, cyan, magenta, black, white
\definecolor{mygreen}{RGB}{28,172,0} % color values Red, Green, Blue
\definecolor{mylilas}{RGB}{170,55,241}
\usepackage{fancyhdr}
\pagestyle{fancyplain}
\fancyhead{}
\renewcommand{\headrulewidth}{0pt} % Remove header underlines
\fancyfoot[L]{} % Empty left footer
\fancyfoot[C]{} % Empty center footer
\fancyfoot[R]{\thepage} 
\usepackage{tikz}
\usetikzlibrary{shapes.geometric,arrows}


\usepackage{sectsty} % Allows customizing section commands
\sectionfont{\centering\normalsize\textbf}
\subsectionfont{\flushleft\normalsize\normalfont\textit}
%\allsectionsfont{\centering} % Make all sections centered

\setlength\parindent{0pt} % remove all indentations in document

%----------------------------------------------------------------------------------------
%	BEGIN DOCUMENT
%----------------------------------------------------------------------------------------
\newcommand{\horrule}[1]{\rule{\linewidth}{#1}}

\begin{document}

		\title{\normalfont \normalsize
			\textsc{University of Witwatersrand, Department of Electrical Engineering} \\ [10pt]
			\horrule{0.5pt} \\ [10pt]
			\huge ELEN4009 Shopping Route Recommender \\
			\horrule{2pt} \\ [10pt]}
		\author{\textbf{\normalsize{Luka Cakic (671913), Ronen Freeman (386910) Devin Taylor (603956) and Matthew Marsden (609293)}} \\ [10pt]}
		\date {\normalsize \today}
		
		\maketitle
		
%\makeatletter
%\renewcommand{\maketitle}{\bgroup\setlength{\parindent}{0pt}
%	\begin{flushleft}
%		\textbf{\@title}
%		
%		\@author\vspace{1mm}
%		
%%		\textit{\small{School of Electrical \& Information Engineering, University of the Witwatersrand, Private Bag 3, 2050, Johannesburg, South Africa}} \vspace{10mm} 
%	\end{flushleft}\egroup
%}
%\makeatother
%
%		\twocolumn[
%		\begin{@twocolumnfalse}
%			\title{\normalsize\textbf{ELEN4009: SOFTWARE ENGINEERING}\\[10pt]{PROJECT PROPOSAL - SHOPPING ROUTE RECOMMENDER}\\[10pt]}
%			\author{\textbf{\normalsize{Luka Cakic (671913), Ronen Freeman (386910) Devin Taylor (603956) and Matthew Marsden (609293)}}}
%			\maketitle
%			
%		\end{@twocolumnfalse} ]
%	
%\onecolumn
		
	\section{Project Definition}
		
		The Shopping Route Recommender is a system that allows a shopper to determine the nearest shops which stock items from his/her shopping list. The shopper adds items to his/her shopping list. Once the shopper has added all required items he/she can run the program to determine a route to follow to get to all relevant shops such that all items on his/her list can be obtained. The program can be set to follow the shortest route, shortest route with minimal total expenses or shortest travel time which allows the shopper to obtain all items on his/her list.
				
	\section{Front End}
		
		The front end or the GUI of the application will be the user interface of the system, and therefore the means through which the user will interact with the program. It will be user friendly, simple to use and flexible. The user will enter his or her items manually one at a time. Options will be given to import a .csv file. Furthermore, the application will give options to save collections of items as a shopping list. Loading of a saved shopping list will be a third method of system input. Further options will be displayed such that the user may specify their preferable means of optimisation, these being: shortest route, minimum travel time and least total expenses. Following the user selection of a 'Generate Route' button, the system will output the optimised route depending on the user selections. This route will be displayed to the user via two methods, one being a list of directions and the other a generated map highlighting the optimised route. Alternative route options will be displayed in a grey route marking colour scheme.  
	
	\section{Back End}
	
	The Back End consists of all the application processing. The main feature of the Back End will be the database. A Relational Database Management System (RDBMS) will be used to store all appropriate information. This type of database is based on a relational model such that tables and fields are interconnected in a logical and sequential manner. Structured Query Language (SQL) is a language that will be used in the application to handle the information stored in the RDBMS. \\
	
	Based on the input of the user, this information will be sent to the Back End to be processed by the SQL. The database will contain a number of different tables. The tables will include headings such as Items,  Shops and Location. The tables will contain their appropriate fields. The relational database scheme will be used to link the tables and fields accordingly. The SQL language will pull out the appropriate information from the tables to be placed in Google Maps. Google Maps will be used to generate a number of routes. Finally based on the preferred optimisation type specified by the user, the best route will be displayed on the GUI map display, along with alternate route options differing in optimisation requirements. 
	
	\section{Group Responsibilities}
	
	The final project will be sectioned into two parts. One group pair will work on and perfect the Front End of the application, the other group pair will design the Back End of the overall system. The two Ends will be combined to form the overall user friendly Shopping Route Recommender Program. 
	

%----------------------------------------------------------------------------------------
%	REFERENCES
%----------------------------------------------------------------------------------------
	
		
		
\end{document}