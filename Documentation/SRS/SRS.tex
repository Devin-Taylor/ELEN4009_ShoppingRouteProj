\documentclass[10pt, a4paper, onecolumn]{scrartcl}
\usepackage{cite}  
\usepackage{times}
\usepackage{amsmath}
\usepackage{amsfonts}
\usepackage{amssymb}
\usepackage{graphicx}
\usepackage{listings}
\usepackage{enumitem} % used for list - no spaces between items
\usepackage[english]{babel} % English language/hyphenation
\usepackage[top=2cm, bottom= 3.2cm, left=2cm, right=2cm, columnsep=0.6cm]{geometry}
\usepackage{color} %red, green, blue, yellow, cyan, magenta, black, white
\definecolor{mygreen}{RGB}{28,172,0} % color values Red, Green, Blue
\definecolor{mylilas}{RGB}{170,55,241}
\usepackage{fancyhdr}
\pagestyle{fancyplain}
\fancyhead{}
\renewcommand{\headrulewidth}{0pt} % Remove header underlines
\fancyfoot[L]{} % Empty left footer
\fancyfoot[C]{} % Empty center footer
\fancyfoot[R]{\thepage} 
\usepackage{tikz}
\usetikzlibrary{shapes.geometric,arrows}

\usepackage{sectsty} % Allows customizing section commands
\sectionfont{\centering\normalsize\textbf}
\subsectionfont{\flushleft\normalsize\normalfont\textit}
%\allsectionsfont{\centering} % Make all sections centered

\setlength\parindent{0pt} % remove all indentations in document

%----------------------------------------------------------------------------------------
%	BEGIN DOCUMENT
%----------------------------------------------------------------------------------------
\newcommand{\horrule}[1]{\rule{\linewidth}{#1}}

\begin{document}
	
	\title{\normalfont \normalsize
		\textsc{University of Witwatersrand, Department of Electrical Engineering} \\ [10pt]
		\horrule{0.5pt} \\ [10pt]
		\huge Software Requirement Specification for Shopping Route Recommender \\
		\horrule{2pt} \\ [10pt]}
	\author{\textbf{\normalsize{Luka Cakic (671913), Ronen Freeman (386910), Devin Taylor (603956) and Matthew Marsden (609293)}} \\ [10pt]}
	\date {\normalsize \today}
	
	\maketitle
	
	
	\section{Introduction}
	
		\subsection{Purpose}
		
		\subsection{Document Conventions}
		
		\subsection{Intended Audience and Reading Suggestions}
		
		\subsection{Project Scope}
		
		\subsection{References}
	
	\section{Overall Description}
	
		\subsection{Product Perspective}
		
			The Shopping Route Recommender is an application used by consumers to maximise their shopping experience in terms of three preferred optimisations: minimum cost, travel distance and travel time. The consumer is able to log onto a Website or Smartphone application and create a shopping list with a desired route being generated.  Enabling a user to optimise their shopping experience is a potential success from the start, as their daily routines can become more efficiently and effectively undertaken. The application's use is not only restricted to the general public, but can also be used by businesses and companies involved in the stock collection courier service industries. The application is aimed at being user friendly, simple, and interactive with maximum customisation being a priority aspect in order to maximise an individuals needs. 
		
		\subsection{Product Features}
		
			The list of product features below aim to provide an easy-to-use, customizable application interface for all users. 
		
			\begin{itemize}
				\item Interactive shopping list menu.
				\begin{itemize}
					\item add or remove item
				\end{itemize}
				\item Interactive optimisation selection options.
				\begin{itemize}
					\item minimise cost
					\item minimise travel time
					\item minimise travle distance
				\end{itemize}
				\item Interactive shopping area selection options.
				\begin{itemize}
					\item select from a number of suburbs or regions
				\end{itemize}
				\item Interactive route map displaying alternate routes for selection.
				\begin{itemize}
					\item rotate map
					\item slide map
					\item zoom in/out
					\item satellite view
				\end{itemize}
			\end{itemize}
		
		\subsection{User Classes and Characteristics}
		
			The application is aimed for the general public's use as well as certain business industries. 
			
			\begin{itemize}
				\item General Public
				\begin{itemize}
					\item General population wanting to buy their routine shopping list
					\item General population looking for more specific products and their preferred optimised route
					\item Foreign individuals looking for their ideal shopping locations or travel routes
				\end{itemize}
				\item Business Industries
				\begin{itemize}
					\item Courier companies collecting stock or products from various distributors/stores
				\end{itemize}
			\end{itemize}
		
		\subsection{Operating Environment}
		
			Shopping Route Recommender is an application designed to run on the most Web Browsers as well on Google Android and Mac OS X Smartphones. 
			
			\begin{itemize}
				\item Software Requirements
				\begin{itemize}
					\item Internet connectivity
					\item Entry level Smartphone
					\item Mozilla Firefox, Microsoft Edge, Google Chrome, Microsoft Explorer, Safari
				\end{itemize}
				\item Hardware Requirements
				\begin{itemize}
					\item Entry level Smartphone with interactive touch screen
				\end{itemize}
			\end{itemize}
		
		\subsection{Design and Implementation Constraints}
		
			Shopping Route Recommender is platform independent and is written in \textcolor{red}{language}. In addition, Google Maps API is implemented for generating the desired optimised shopping route. The accuracy of the generated route and optimisation algorithms is therefore dependent on the accuracy of the Google Maps detailing. 
		
		\subsection{User Documentation}
		
			A general help and FAQ menu will be provided within the application. This will function as the "user manual" of the application. 
		
		\subsection{Assumptions and Dependencies}
	
	\section{System Features}
	
		\subsection{System Feature 1 - }
		
		\subsection{System Feature 2 - }
		
		\subsection{System Feature 3 - }
	
	\section{External Interface Requirements}
	
		\subsection{User Interfaces - GUI}
		
		\subsection{Hardware Interfaces}
		
		\subsection{Software Interfaces}
		
		\subsection{Communications Interfaces}
	
	\section{Other Non-functional Requirements}
	
		\subsection{Performance Requirements}
		
		\subsection{Safety Requirements}
		
		\subsection{Security Requirements}
		
		\subsection{Software Quality Attributes}
		
		\subsection{Other Requirements}
	
	\begin{thebibliography}{99}
		\bibitem{probs}Fluid Switch. \textit{Airplane Fule Gauges: How they Work, Challenges, \& Solutions}. Fluid Switch. URL: http://www.fluidswitch.com/blog/airplane-fuel-gauges/. [Accessed: February 15, 2016]
	\end{thebibliography}
	
	
	%----------------------------------------------------------------------------------------
	%	REFERENCES
	%----------------------------------------------------------------------------------------
	
	
	
\end{document}